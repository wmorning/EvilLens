\begin{document}

In order to correctly model a gravitational lensing system, one 
needs to know the path that light rays take through a space to reach an observer.
Because only a small portion of the emitted rays ever reach an observer,
a significant amount of time is saved when modeling a lens by tracing
light rays {\i backwards} rather than forwards.  To make an image this
way, all one needs to do is form an image (which we take to be a
grid of pixels with uniform spacing), and trace the path of light rays
from an image pixel back to their origin.  In a gravitational lens system
many rays are deflected by some amount, and thus most originate
from a different angular position than their apparent position in the image 
plane.  Typically, the apparent position on the image plane is referred to 
as $\vec{\theta}$, while the true position in the source plane is referred to
as $\vec{\beta}$.  In this work, we will often refer to these positions with 
their "x" and "y" components for convenience, but it is important to note
that this x and y only refer to two orthogonal dimensions on the image or source 
plane (rather than preferred directions).  The relation between the positions 
$\vec{\theta}$ and $\vec{\beta}}$ are related by the deflection angle, which 
is traditionally denoted as $\vec{\alpha}$, by the following relation:
\begin{equation}  \vec{\beta} = \vec{\theta}-\vec{\alpha}   \end{equation}
Which is referred to as the "lens equation."

If $\vec{\alpha}$ is known, then each point in an image is traced backwards
to a corresponding point in the source plane.  However, the major complication is that
$\vec{\alpha}$ is almost never known {\i a priori}.  Instead, we model the 
mass distribution of a lens and predict the 
deflection angles using the equations for bending of light that fall out of 
general relativity.  The process by which this is done is as follows:

\begin{enumerate}
\item Compared to the distances between objects, their thicknesses is very small.  We thus 
approximate them as infinitesimally thin sheets with a surface density $\Sigma$, which is found by 
integrating the three dimensional density along the observing axis.
\item The surface density can be made dimensionless by dividing by the critical density $\Sigma_{c}$,
this quantity is called the convergence and is denoted using the letter $\kappa$.
\item For a given point in an image, weight the convergence by its distance from 
that point.  Perform the double integral of this weighted function.  The resulting 
value is the deflection angle $\vec{\alpha}$ at that point.
\item Repeat These steps for each point in your image to get a full map of the 
deflection angles. 
\end{enumerate}

Typically, this process is done for idealized density profiles, which have analytic
solutions.  However, our implementation is instead to perform this process
numerically, thus freeing us from the assumption of idealized lensing mass 
distributions.  To do this, we create a numerical grid of the convergence, and
numerically integrate the weighted grid using simpsons rule in two dimensions.
Of course, integrating over this map using a finite numerical resolution 
leads to numerical errors due to the finite pixel size.  Additionally, if
integrating a mass distribution that spans all of space, integrating over a finite 
grid will systematically underrepresent the mass located exterior to the grid, and 
thus will systematically overestimate the magnitude deflection angles.
To deal with these two problems, we have incorporated padding into our code, meaning 
that we use a larger grid of points to represent our mass map, and integrate for 
the deflection angles in only a small portion of the map.  This yields a significant
improvement in the accuracy of our estimated deflection angles, without oversampling 
image regions significantly outside of the critical radius (in which multiple images
would not be produced anyways).  Even this becomes computationally intensive however,
as one ideally needs to achieve a pixel resolution on the order of milliarcseconds as 
well as span a mass map that covers nearly twenty arcseconds on an axis (to supress
any biases in the calculated deflection angles).  Thus it is most productive to find 
a balance between time and accuracy.  We have thus tabulated a grid of pixel sizes, 
padding factor (n), number of mass pixels, run time ($t_{run}$), and percent rms error (\epsilon) in the 
measured mass of a subhalo.

\begin{table}[tb]
\caption{Accuracy vs. Time Testing Results}
\label{tab:Acc}
\begin{center}
\begin{tabular}{lllll}
\tableline \tableline
\# of $\kappa$ Pixels & Pixel Size & n & $t_{run}$ & $\epsilon$ \\
 ~ & $(arcsec)$ & ~ & seconds & $(\%)$ \\
 100 & 0.02 & 1\\
 200 & 0.02 & 2\\
 400 & 0.02 & 4 & 191 & 5\\
 400 & 0.02 & 2\\
 800 & 0.02 & 8\\
 800 & 0.02 & 4\\
 800 & 0.02 & 2\\
 1600 & 0.02 & 16\\
 1600 & 0.02 & 8\\

\tableline{}
\end{tabular}
\end{center}
\tablecomments{Accuracy and Run-time in calculating the deflection angles using EvilLens.  In all 
cases, we used a $\kappa$ map consisting of two SIE lenses; a main halo with velocity dispersion
$\sigma_{v}=200~km~s^{-1}$ and axis ratio $q=0.75$, and a subhalo with $\sigma_{v}=40.0~km~s^{-1}$ 
and axis ratio q=0.99 .  Our proxy for accuracy is the error in the measured subhalo mass.}

\end{document}
