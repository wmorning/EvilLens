\begin{document}
In order to properly produce a lensed image (given a source image, and a 
lensing mass distribution), one needs to trace the light rays backwards 
from the observed image to the source.  When a light ray is deflected by an 
object, the change in angle by which it is deflected \alpha can be inferred
using a projection of the lensing mass density onto a 2-dimensional plane.
Mathematically, this is: 

\begin{equation} \alpha = \int\int \kappa(x,y) \frac{}{}
\end{equation}
Where \alpha is the deflection angle, and \kappa(x,y) is the dimensionless
surface density

\begin{equation} \kappa(x,y) = \frac{\Sigma(x,y)}{\Sigma_{c}}

\end{equation}

Here, $\Sigma_{c}$ is the critical surface density, which is dependent on the
lens geometry.

\begin{equation} \Sigma_{c} = \frac{c^2}{4\pi G}\frac{D_{s}}{D_{d}D_{ds}}
\end{equation}

Given a specific lensing geometry, and a specific mass profile, a map of 
\kappa can be produced over the lensing plane, and integrating over that map
produces a map of the observing pixels in the source frame.

\end{document}
