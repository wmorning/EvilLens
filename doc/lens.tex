\begin{document}
In order to properly produce a lensed image (given a source image, and a 
lensing mass distribution), one needs to trace the light rays backwards 
from the observed image to the source.  When a light ray is deflected by an 
object, the change in angle by which it is deflected \alpha can be inferred
using a projection of the lensing mass density onto a 2-dimensional plane.
Mathematically, this is: 

\begin{equation} \alpha = \int\int \kappa(x,y) \frac{}{}
\end{equation}
Where \alpha is the deflection angle, and \kappa(x,y) is the dimensionless
surface density

\begin{equation} \kappa(x,y) = \frac{\Sigma(x,y)}{\Sigma_{c}}

\end{equation}

Here, $\Sigma_{c}$ is the critical surface density, which is dependent on the
lens geometry.

\begin{equation} \Sigma_{c} = \frac{c^2}{4\pi G}\frac{D_{s}}{D_{d}D_{ds}}
\end{equation}

Given a specific lensing geometry, and a specific mass profile, a map of 
\kappa can be produced over the lensing plane, and integrating over that map
produces a map of the observing pixels in the source frame.

Of course, integrating over this map using a finite numerical resolution 
leads to numerical errors due to the finite pixel size.  Additionally, if
integrating a mass distribution that spans all of space, integrating over a finite 
grid will systematically underrepresent the mass located exterior to the grid.
To deal with these two problems, we have incorporated padding into our code, meaning 
that we use a larger grid of points to represent our mass map, and integrate for 
the deflection angles in only a small portion of the map.  This yields a significant
improvement in the accuracy of our estimated deflection angles, without oversampling 
image regions significantly outside of the critical radius (in which multiple images
would not be produced anyways).  Even this becomes computationally intensive however,
as one ideally needs to achieve a pixel resolution on the order of milliarcseconds as 
well as span a mass map that covers nearly twenty arcseconds on an axis (to supress
any biases in the calculated deflection angles).  Thus it is most productive to find 
a balance between time and accuracy.  We have thus tabulated a grid of pixel sizes, 
padding factor (n), number of mass pixels, run time ($t_{run}$), and percent rms error (\epsilon) in the 
measured mass of a subhalo.

\begin{table}[tb]
\caption{Accuracy vs. Time Testing Results}
\label{tab:Acc}
\begin{center}
\begin{tabular}{lllll}
\tableline \tableline
\# of $\kappa$ Pixels & Pixel Size & n & $t_{run}$ & $\epsilon$ \\
 ~ & $(arcsec)$ & ~ & seconds & $(\%)$ \\
 100 & 0.02 & 1\\
 200 & 0.02 & 2\\
 400 & 0.02 & 4 & 191 & 5\\
 400 & 0.02 & 2\\
 800 & 0.02 & 8\\
 800 & 0.02 & 4\\
 800 & 0.02 & 2\\
 1600 & 0.02 & 16\\
 1600 & 0.02 & 8\\

\tableline{}
\end{tabular}
\end{center}
\tablecomments{Accuracy and Run-time in calculating the deflection angles using EvilLens.  In all 
cases, we used a $\kappa$ map consisting of two SIE lenses; a main halo with velocity dispersion
$\sigma_{v}=200~km~s^{-1}$ and axis ratio $q=0.75$, and a subhalo with $\sigma_{v}=40.0~km~s^{-1}$ 
and axis ratio q=0.99 .  Our proxy for accuracy is the error in the measured subhalo mass.}

\end{document}
